\documentclass[11pt]{paper}
\usepackage[letterpaper]{geometry}

\usepackage{tikz-cd}
\usepackage{amsthm}

%%%%%%%%%%%%%%%%%%%%%%%%%%%%%%%%%%%%%%%%
% Basic packages
%%%%%%%%%%%%%%%%%%%%%%%%%%%%%%%%%%%%%%%%
\usepackage{amsmath,amsthm,amssymb}
\usepackage{mathtools}
\usepackage{etoolbox}
\usepackage{fancyhdr}
\usepackage{xcolor}
\usepackage{hyperref}
\usepackage{xspace}
\usepackage{comment}
\usepackage{url} % for url in bib entries
\usepackage{mathrsfs}


\theoremstyle{remark}
\newtheorem{problem}{Problem}
\newtheorem*{prob}{Problem}
\newtheorem*{solution}{{\bf Solution}}
\newtheorem*{hint}{{\it Hint}}

%%%%%%%%%%%%%%%%%%%%%%%%%%%%%%%%%%%%%%%%%%%%%%%%%%
%% Surround the problem and solution with 
%% \begin{ProbBox}  and   \end{ProbBox}
%% to prevent pagebreaks.
\newenvironment{ProbBox}{\noindent\begin{minipage}{\linewidth}}{\end{minipage}}

%%%%%%%%%%%%%%%%%%%%%%%%%%%%%%%%%%%%%%%%
% Acronyms
%%%%%%%%%%%%%%%%%%%%%%%%%%%%%%%%%%%%%%%%
\usepackage[acronym, shortcuts]{glossaries}

%% HERE IS HOW YOU DEFINE ACRONYMS:
\newacronym{FTA}{FTA}{Fundamental Theorem of Algebra}
\newacronym{CRT}{CRT}{Chinese Remainder Theorem}

% Make \ac robust.
\robustify{\ac}

%%%%%%%%%%%%%%%%%%%%%%%%%%%%%%%%%%%%%%%%
% Fancy page style
%%%%%%%%%%%%%%%%%%%%%%%%%%%%%%%%%%%%%%%%
\pagestyle{fancy}
\newcommand{\metadata}[2]{
  \lhead{}
  \chead{}
  \rhead{\bfseries Math 700: Linear Algebra}
  \lfoot{#1}
  \cfoot{#2}
  \rfoot{\thepage}
}
\renewcommand{\headrulewidth}{0.4pt}
\renewcommand{\footrulewidth}{0.4pt}


\newrobustcmd*{\vocab}[1]{\emph{#1}}
\newrobustcmd*{\latin}[1]{\textit{#1}}

%%%%%%%%%%%%%%%%%%%%%%%%%%%%%%%%%%%%%%%%
% Customize list enviroonments
%%%%%%%%%%%%%%%%%%%%%%%%%%%%%%%%%%%%%%%%
% package to customize three basic list environments: enumerate, itemize and description.
\usepackage{enumitem}
\setitemize{noitemsep, topsep=0pt, leftmargin=*}
\setenumerate{noitemsep, topsep=0pt, leftmargin=*}
\setdescription{noitemsep, topsep=0pt, leftmargin=*}

%%%%%%%%%%%%%%%%%%%%%%%%%%%%%%%%%%%%%%%%
%% Space between problems
\newrobustcmd*{\probskip}{\vskip1cm}



%%test This is the Homework LaTeX template.  Use this file to fill in your solutions. 
%%
%% Notes: 
%%    1. Write your answers inside a \begin{solution}...\end{solution} environment.
%%
%%    2. If you will use references, insert bibtex reference entries in the file
%%       Math700.bib.  (Create that file if it doesn't yet exist.)
%%
%%    3. If you will use acronyms, please define them in the macros.tex file.
%%
%%    4. Please try to check that your file compiles:
%%
%%       Mac OS X users: you might try MacTeX. 
%%       Windows users: you might try proTeXt. 
%%       Linux users: most come with TeX; otherwise do a full install of TeXLive.
%%
%%       There is a Makefile in this directory, so on Linux you could just 
%%       enter `make` to compile all the Homework*.tex files at once.
%%
%%    5. Please don't hesitate to inform the prof if you have trouble, or open
%%       a ``New issue'' or create a new ``Wiki page'' on GitHub.  Otherwise,
%%       send an email to williamdemeo@gmail.com.
%%
%%    6. It will probably be hard to keep everyone's notation consistent.
%%       For the most basic symbols, we should have some conventions and use
%%       LaTeX macros to keep the conventions consistent and easy to remember.
%%       For example, to denote an algebra,
         \newcommand\alg[1]{\ensuremath{\mathbf{#1}}}
         \newcommand{\<}{\ensuremath{\langle}}
         \renewcommand{\>}{\ensuremath{\rangle}}
%%       So, an algebra in LaTeX is typed as $\alg{A} = \<A, F\>$.
%%       Similarly, for a field, let's use:
         \newcommand\fld[1]{\ensuremath{\mathbb{#1}}}
%%       So, a field in LaTeX is typed as $\fld{F}$.

%%
%%    7. Replace these names with yours!!!
         \metadata{Danny and Heather}{Homework 1 -- 2014/02/03}
         \author{Danny Rorabaugh and Heather Smith}
%%
%%    8. Update the title and date as appropriate.
         \title{Homework 1}
         \date{03 February 2014}

\begin{document}

\maketitle


%%%%%%%%%%%%%%%%%%%%%%%%%%%%%%%%%%%%%%%%%%%%%%%%%%%%%%%%%%%%%%%%%%%%%%%%%%%%
\begin{problem}[Golan 12]
For a field $\fld{F} = \<F,+,\cdot, -, 0, 1\>$, 
show that the function $a \mapsto a^{-1}$ is a 
permutation of the set $F \setminus \{0_F\}$.
\end{problem}
\smallskip
\begin{solution}

In order to show that $f:F\setminus \{0_F\}\rightarrow F\setminus \{0_F\}$, $f(a)=a^{-1}$ is a permutation, we prove that $f$ is a well-defined bijection. 

Well-defined: For each $a\in F\setminus \{0_F\}$, there is an element $x\in F\setminus \{0_F\}$ such that $ax=xa=1$ because $F$ is a field. In other words, $x=a^{-1}$. Now suppose there exists $y\in  F\setminus \{0_F\}$ such that $ay=ya=1$. Then $y = 1y = xay = x1 = xax = 1x = x$. Therefore inverses in $F$ are unique. 

Surjective: For any $b\in F\setminus \{0_F\}$, we have $f(b)=b^{-1}\in F\setminus \{0_f\}$ because $F$ is a field. Observe that $bb^{-1}=b^{-1}b=1$. Therefore, $b$ is the unique inverse of $b^{-1}$. So $f(b^{-1}) = b$. 

Injective: For any $a,b\in F\setminus \{0_F\}$, suppose $f(a) = c = f(b)$ for some $c\in F\setminus \{0_F\}$. Then $ac = 1 =bc $, so $a = a1 = acc^{-1} = 1c^{-1} = bcc^{-1} = b1 = b$.

[Note: ``If $c\neq 0$, then $ac = bc$ implies $a=b$'' is a general property of integral domains. You can alternatively use the fact that every field is an integral domain to prove that $f$ is well-defined and injective.]

\end{solution}
\probskip




%%%%%%%%%%%%%%%%%%%%%%%%%%%%%%%%%%%%%%%%%%%%%%%%%%%%%%%%%%%%%%%%%%%%%%%%%%%%
\begin{problem}[Golan 16]
Let $z_1$, $z_2$, and $z_3$ be complex numbers satisfying 
$|z_i| = 1$ for $i = 1, 2, 3$. Show that 
$|z_1 z_2 + z_1 z_3 + z_2 z_3 | = |z_1 + z_2 + z_3|$.
\end{problem}
\smallskip
\begin{solution}
For each $j \in [3]$, since $|z_i|=1$, note that for some real $\theta_j$,
$$\frac{1}{z_j}  = \frac{1}{e^{i\theta_j}} = e^{-i\theta_j}= \overline{z_j}.$$
Therefore,
\begin{eqnarray*}
|z_1 z_2 + z_1 z_3 + z_2 z_3 | & = & |z_1| \cdot |z_2| \cdot |z_3| \cdot \left|\frac{1}{z_3} + \frac{1}{z_2} + \frac{1}{z_1}\right|\\
& = & 1 \cdot 1 \cdot 1 \cdot | \overline{z_3} + \overline{z_2} + \overline{z_1} | \\
& = &  | \overline{z_1 + z_2 + z_3} | \\
& = & |z_1 + z_2 + z_3 |.
\end{eqnarray*}

\end{solution}
\probskip




%%%%%%%%%%%%%%%%%%%%%%%%%%%%%%%%%%%%%%%%%%%%%%%%%%%%%%%%%%%%%%%%%%%%%%%%%%%%
\begin{problem}[Golan 22 {\it Abel's inequality}] 
Let $z_1, \dots, z_n$ be a list of complex
numbers and, for each $1 \leq k \leq n$, 
let $s_k = \sum_{i=1}^k z_i$. For real numbers
$a_1, \dots, a_n$ satisfying 
$a_1 \geq a_2 \geq \cdots \geq a_n \geq 0$, 
show that
\begin{equation}
\label{eq:Abels}  
\left| \sum_{i=1}^n a_i z_i \right| 
\leq a_1 \left( \max_{1 \leq k \leq n} |s_k|\right).
\end{equation}
\end{problem}
\smallskip
\begin{solution}

Define $s_0:=0$. Observe $a_iz_i = a_is_i - a_is_{i-1}$ for $1\leq i\leq n$. Therefore


\begin{align*}
\left|\sum_{i=1}^n a_iz_i \right| & = \left|\sum_{i=1}^n  a_i s_i - a_i s_{i-1}\right|\\
&= \left|\sum_{i=1}^{n-1} s_i(a_i-a_{i+1}) + s_n a_n \right|\\
&\leq \sum_{i=1}^{n-1} \left|s_i (a_i-a_{i+1})\right| + \left|s_n a_n\right|  \tag*{by the Triangle Inequality}\\
&= \sum_{i=1}^{n-1} \left(a_i-a_{i+1})|s_i\right| + a_n\left|s_n\right| \\
&\leq \sum_{i=1}^{n-1}(a_i-a_{i+1}) \max_{1 \leq k \leq n} |s_k| +  a_n\max_{1 \leq k \leq n} |s_k| \\
&= \left(\sum_{i=1}^{n-1} (a_i-a_{i+1}) + a_n\right) \max_{1 \leq k \leq n} |s_k| \\
&= a_1 \max_{1 \leq k \leq n}  |s_k|. \tag*{by telescoping sums}
\end{align*}
\end{solution}
\probskip



%%%%%%%%%%%%%%%%%%%%%%%%%%%%%%%%%%%%%%%%%%%%%%%%%%%%%%%%%%%%%%%%%%%%%%%%%%%%
\begin{problem}[Golan 24]
If $p$ is a prime positive integer, find all subfields of $GF(p)$.
\end{problem}

\begin{solution}
Since every field is a unital ring, every subfield of $GF(p)$ contains the unit 1. The order of 1 is $p$, since $\underbrace{1 + 1 + \cdots + 1}_{p} = 0$, but $\underbrace{1 + 1 + \cdots + 1}_{k} \neq 0$ for positive $k<p$.

Suppose  $\underbrace{1 + 1 + \cdots + 1}_{a} = \underbrace{1 + 1 + \cdots + 1}_{b}$ for some positive $a,b <p$. Then $a=b$. Thus, any subfield with 1 contains at least $p$ distinct elements. Since $|GF(p)|=p$, it is itself the only subfield.

\end{solution}
%\probskip
\newpage

%%%%%%%%%%%%%%%%%%%%%%%%%%%%%%%%%%%%%%%%%%%%%%%%%%%%%%%%%%%%%%%%%%%%%%%%%%%%
\begin{problem}
Write down the definition of a \emph{module} as a (universal) algebra, 
$\alg{M} = \< M, F\>$.  That is, describe the set $F$ of operations and 
give the conditions that they should satisfy in order for $\alg{M}$ to 
agree with the classical definition of a module over a ring.\\[4pt]
[{\it Hint:} Let $\alg{R} = \<R, +, \cdot, -, 0, 1\>$ be a ring and, for each $r\in R$, define a scalar multiply operation $f_r \in F$.]
\end{problem}

\begin{solution}
Let $\alg{A} = \< V, +, -, 0\>$ be an Abelian group. 
Let $\alg{R}$ be the unital ring 
\[\alg{R} = \<R, +, \cdot, -, 0, 1\>\]
Define module $\alg{V}$ as follows: 
\[\alg{V}=\<V, +, -, 0, \{f_r: r\in R\}\>\] 
where the addition, additive inverse, and zero are defined on $V$ as they are in the Abelian group $\alg{A}$.
Each $f_r$ is a function $f_r: V\rightarrow V$ so that for any $r, r_1, r_2\in R$ and any $v, v_1, v_2\in V$ each of the following is satisfied: 
	\begin{itemize}
	\item $f_r(v_1+v_2)=f_r(v_1) + f_r(v_2)$
	\item $f_{r_1+r_2}(v) = f_{r_1}(v) + f_{r_2}(v)$
	\item $f_{r_1}\left(f_{r_2}(v)\right) = f_{r_1r_2} (v)$
	\item $f_1(v) = v$
	\end{itemize}
\end{solution}
\probskip



%%%%%%%%%%%%%%%%%%%%%%%%%%%%%%%%%%%%%%%%%%%%%%%%%%%%%%%%%%%%%%%%%%%%%%%%%%%%
\begin{problem}
Let $\alg{R} = \<R, +, -, \cdot, 0, 1\>$ be a ring.  
\begin{enumerate}
\item Define \emph{left ideal} of $\alg{R}$.
\item Let $\mathscr{A} = \{A_i : i \in \mathscr{I}\}$ be a family of left ideals
of $\alg{R}$.  Prove that $\bigcap \mathscr{A}$ is a left ideal.
\end{enumerate}
\end{problem}
\smallskip
\begin{solution} 
$\left.\right.$\\
\begin{enumerate}
\item A left ideal $I$ of $\alg{R}$ is a subset of $R$ which forms a subgroup under addition and for any $a\in I$ and $r\in R$, we have $ra\in I$. 
\item Clearly $\bigcap \mathscr{A}$ is a subset of $R$ since each $A_i\in \mathscr{A}$ is a subset of $R$. Let $a,b\in \bigcap \mathscr{A}$ and $r\in R$. By definition of intersection, $a,b\in A_i$ for all $A_i \in \mathscr{A}$. Therefore $ra, a+b\in A_i$ for all $A_i \in \mathscr{A}$ because each $A_i$ is a left ideal. But this implies that $ra,a+b\in \bigcap \mathscr{A}$, proving that $\bigcap \mathscr{A}$ is a left ideal. 

\end{enumerate}
\end{solution}
%\probskip
\newpage 

%%%%%%%%%%%%%%%%%%%%%%%%%%%%%%%%%%%%%%%%%%%%%%%%%%%%%%%%%%%%%%%%%%%%%%%%%%%%
\begin{problem}
Let $\alg{R}$ be a ring and fix $a, b \in R$.  Prove that if $1 - ba$ is left
invertible, then $1 - ab$ is also left invertible.  What is the inverse?\\[4pt]
[{\it Hint:} Consider the left ideal $R(1 - ab)$.  It contains the left ideal $Rb(1-ab) = Rb$ and therefore contains 1. Verify these statements, then try to compute the inverse of $1-ab$. (Ask for more hints as needed.)] 
\end{problem}
\smallskip
\begin{solution}
We can show existence as follows: 
Observe $Rb(1-ab)\subseteq R(1-ab)$ since $Rb\subseteq R$.  Because $(1-ba)$ is left invertible, $R = R(1-ba)$ and
\[Rb=R(1-ba)b=R(b-bab)=Rb(1-ab).\] Therefore $Rb\subseteq R(1-ab)$. Observe $ab\in Rb$ and $(1-ab)\in R(1-ab)$ so \[1=(1-ab)+ab\in R(1-ab),\] which implies that $(1-ab)$ is invertible.


%[{\it Danny Hint:} Show that $b \in R(1-ab)$ and therefore $ab \in R(1-ab)$. Then try to compute the left inverse of $1-ab$.] 
To explicitly find the left inverse of $(1-ab)$, let $c \in R$ be such that $c(1-ba) = 1$. Then
\begin{eqnarray*}
1 & = & 1 - ab + ab\\
& = & 1 - ab + a1b\\
& = & 1 - ab + ac(1-ba)b\\
& = & (1 - ab) + ac(b-bab)\\
& = & 1(1 - ab) + acb(1-ab)\\
& = & (1+acb)(1-ab)
\end{eqnarray*}

Therefore, the left inverse of $(1-ab)$ is $(1+acb)$, where $c$ is the left inverse of $(1-ba)$.



\end{solution}
\probskip


%% If you will use references, add your refs to the Math700.bib file.
%% and then uncomment the following lines.
%% \bibliographystyle{plain}
%% \bibliography{Math700}

\end{document}
