\documentclass[11pt]{paper}
\usepackage[letterpaper]{geometry}

\usepackage{tikz-cd}
\usepackage{amsthm}

%%%%%%%%%%%%%%%%%%%%%%%%%%%%%%%%%%%%%%%%
% Basic packages
%%%%%%%%%%%%%%%%%%%%%%%%%%%%%%%%%%%%%%%%
\usepackage{amsmath,amsthm,amssymb}
\usepackage{mathtools}
\usepackage{etoolbox}
\usepackage{fancyhdr}
\usepackage{xcolor}
\usepackage{hyperref}
\usepackage{xspace}
\usepackage{comment}
\usepackage{url} % for url in bib entries
\usepackage{mathrsfs}


\theoremstyle{remark}
\newtheorem{problem}{Problem}
\newtheorem*{prob}{Problem}
\newtheorem*{solution}{{\bf Solution}}
\newtheorem*{hint}{{\it Hint}}

%%%%%%%%%%%%%%%%%%%%%%%%%%%%%%%%%%%%%%%%%%%%%%%%%%
%% Surround the problem and solution with 
%% \begin{ProbBox}  and   \end{ProbBox}
%% to prevent pagebreaks.
\newenvironment{ProbBox}{\noindent\begin{minipage}{\linewidth}}{\end{minipage}}

%%%%%%%%%%%%%%%%%%%%%%%%%%%%%%%%%%%%%%%%
% Acronyms
%%%%%%%%%%%%%%%%%%%%%%%%%%%%%%%%%%%%%%%%
\usepackage[acronym, shortcuts]{glossaries}

%% HERE IS HOW YOU DEFINE ACRONYMS:
\newacronym{FTA}{FTA}{Fundamental Theorem of Algebra}
\newacronym{CRT}{CRT}{Chinese Remainder Theorem}

% Make \ac robust.
\robustify{\ac}

%%%%%%%%%%%%%%%%%%%%%%%%%%%%%%%%%%%%%%%%
% Fancy page style
%%%%%%%%%%%%%%%%%%%%%%%%%%%%%%%%%%%%%%%%
\pagestyle{fancy}
\newcommand{\metadata}[2]{
  \lhead{}
  \chead{}
  \rhead{\bfseries Math 700: Linear Algebra}
  \lfoot{#1}
  \cfoot{#2}
  \rfoot{\thepage}
}
\renewcommand{\headrulewidth}{0.4pt}
\renewcommand{\footrulewidth}{0.4pt}


\newrobustcmd*{\vocab}[1]{\emph{#1}}
\newrobustcmd*{\latin}[1]{\textit{#1}}

%%%%%%%%%%%%%%%%%%%%%%%%%%%%%%%%%%%%%%%%
% Customize list enviroonments
%%%%%%%%%%%%%%%%%%%%%%%%%%%%%%%%%%%%%%%%
% package to customize three basic list environments: enumerate, itemize and description.
\usepackage{enumitem}
\setitemize{noitemsep, topsep=0pt, leftmargin=*}
\setenumerate{noitemsep, topsep=0pt, leftmargin=*}
\setdescription{noitemsep, topsep=0pt, leftmargin=*}

%%%%%%%%%%%%%%%%%%%%%%%%%%%%%%%%%%%%%%%%
%% Space between problems
\newrobustcmd*{\probskip}{\vskip1cm}



%%test This is the Homework LaTeX template.  Use this file to fill in your solutions. 
%%
%% Notes: 
%%    1. Write your answers inside a \begin{solution}...\end{solution} environment.
%%
%%    2. If you will use references, insert bibtex reference entries in the file
%%       Math700.bib.  (Create that file if it doesn't yet exist.)
%%
%%    3. If you will use acronyms, please define them in the macros.tex file.
%%
%%    4. Please try to check that your file compiles:
%%
%%       Mac OS X users: you might try MacTeX. 
%%       Windows users: you might try proTeXt. 
%%       Linux users: most come with TeX; otherwise do a full install of TeXLive.
%%
%%       There is a Makefile in this directory, so on Linux you could just 
%%       enter `make` to compile all the Homework*.tex files at once.
%%
%%    5. Please don't hesitate to inform the prof if you have trouble, or open
%%       a ``New issue'' or create a new ``Wiki page'' on GitHub.  Otherwise,
%%       send an email to williamdemeo@gmail.com.
%%
%%    6. It will probably be hard to keep everyone's notation consistent.
%%       For the most basic symbols, we should have some conventions and use
%%       LaTeX macros to keep the conventions consistent and easy to remember.
%%       For example, to denote an algebra,
         \newcommand\alg[1]{\ensuremath{\mathbf{#1}}}
         \newcommand{\<}{\ensuremath{\langle}}
         \renewcommand{\>}{\ensuremath{\rangle}}
%%       So, an algebra in LaTeX is typed as $\alg{A} = \<A, F\>$.
%%       Similarly, for a field, let's use:
         \newcommand\fld[1]{\ensuremath{\mathbb{#1}}}
%%       So, a field in LaTeX is typed as $\fld{F}$.

%%
%%    7. Replace these names with yours!!!
         \metadata{}{Homework 1 hints -- 2014/01/13}
         \author{}
%%
%%    8. Update the title and date as appropriate.
         \title{Homework 1 hints}
         \date{January 30, 2014}

\begin{document}

\maketitle

You need not make use of these hints. There may be simpler ways to solve
the problems.

%%%%%%%%%%%%%%%%%%%%%%%%%%%%%%%%%%%%%%%%%%%%%%%%%%%%%%%%%%%%%%%%%%%%%%%%%%%%
\begin{prob}[Golan 16]
Let $z_1$, $z_2$, and $z_3$ be complex numbers satisfying 
$|z_i| = 1$ for $i = 1, 2, 3$. Show that 
$|z_1 z_2 + z_1 z_3 + z_2 z_3 | = |z_1 + z_2 + z_3|$.
\end{prob}

\smallskip
\begin{hint}
Let $w =  z_1 z_2 z_3$. Note that the modulus of $w$ is $|w| = 1$, since each $z_i$ has modulus 1.
The left hand side of the equality we want to establish is
$| w/z_3 + w/z_2 + w/z_1 |$.
Factoring out $|w|$, we have $|w| |1/z_3 + 1/z_2 + 1/z_1|$.
Note that since $|z_i|=1$, the conjugate of $z_i$ is $1/z_i$.
\end{hint}
\probskip

%%%%%%%%%%%%%%%%%%%%%%%%%%%%%%%%%%%%%%%%%%%%%%%%%%%%%%%%%%%%%%%%%%%%%%%%%%%%
\begin{prob}[Golan 22 {\it Abel's inequality}] 
Let $z_1, \dots, z_n$ be a list of complex
numbers and, for each $1 \leq k \leq n$, 
let $s_k = \sum_{i=1}^k z_i$. For real numbers
$a_1, \dots, a_n$ satisfying 
$a_1 \geq a_2 \geq \cdots \geq a_n \geq 0$, 
show that
\begin{equation}
\label{eq:Abels}  
\left| \sum_{i=1}^n a_i z_i \right| 
\leq a_1 \left( \max_{1 \leq k \leq n} |s_k|\right).
\end{equation}
\end{prob}
\smallskip
\begin{hint}
The $k$th term of $\sum a_i z_i$ is 
$a_k z_k = a_k (z_1+\cdots + z_k) - a_k (z_1+\cdots + z_{k-1})$,
so
\begin{align*}
\left| \sum_{i=1}^n a_i z_i \right|  
&= \left|a_n (z_1+\cdots + z_n) - a_n (z_1+\cdots + z_{n-1}) \right.\\
&+ a_{n-1} (z_1+\cdots + z_{n-1}) - a_{n-1} (z_1+\cdots + z_{n-2}) \\
& \vdots\\
&+ a_2 (z_1+z_2) - a_2 z_1 + \left.a_1 z_1\right|.
\end{align*}
Collecting like terms in the expression on the right, we have
\[
\left| z_1(a_1-a_2)+ (z_1+ z_2)(a_2-a_3) + \cdots + (z_1+\cdots +
  z_{n-1})(a_{n-1}-a_n) + (z_1+\cdots + z_{n})a_n \right|. \]
Consider bounding this expression by the maximum of its terms 
(maximizing over both $j$ and $k$).  Finally, note that $a_j - a_{j+1} < a_1$.
Use these facts, filling in any missing details, to establish the claim.  
\end{hint}
\probskip


\end{document}
